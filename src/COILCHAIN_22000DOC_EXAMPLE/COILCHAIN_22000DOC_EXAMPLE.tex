% Save with file name COILCHAIN-YY####TYPE\_desciption.tex
% mkdir COILCHAIN-YY####TYPE\_description/ if doc as dependencies
% /img/   for image    /chap/    for chapters    /fig/    for figure sources

\documentclass[technicalreport]{coilchain}

\usepackage{multirow}
\usepackage{tabularx}

\begin{document}

% increment sequence according to repository
\docid{
code = COILCHAIN,
sequence = 0000,
subcode = ,
year = 2022,
type = documentation,
revision = 0.1}

\title{Manual for the coilchain documentation}

\author{Benjamin Bonnal}

% DD MM YYYY
\date{08}{06}{2022}


\maketitle
	
	\begin{introduction}
Communication is key to a performing team. In  addition to having a good 
communication, a team must learn from previous mistakes and keep a trace of 
achieved work. Instant messaging services excels at allowing fast communication 
and quick discussion. However, they are not suited to long explanation and 
in-depth concept discussion. Furthermore, they are not well suited for storage 
of the past discussion (the free Slack edition only keeps the 10\,000 most 
recent messages).

This document presents the \texttt{coilchain} \LaTeX document class, a 
standardized document format and classification system to ease communication 
and storage of documents.

To start using \verb|coilchain| as a document class, simply write the 
following line at the top of your document: \verb|\documentclass{coilchain}|.
	\end{introduction}
	
	\tableofcontents
	
	\section{Document Identifier}
A document identifier is used to uniquely and unambiguously identify a 
document. Such a number may later be used in an Enterprise Content Management 
software (ECM). The document identifier is formed with a prefix code, the 
current year and a sequence number in the current year.

For example, a document is identified by:

\begin{center}
	\texttt{COILCHAIN\_220000DOC\_TITLE}
\end{center}

\texttt{COILCHAIN} is the prefix code. All coilchain documents should start with 
this prefix code. \texttt{22} is the current year. \texttt{0000} is the 
sequence number. This document is the first document using this identification 
system, thus it is the document number zero. \texttt{DOC} is the document type: 
\texttt{DOC}umentation.

The document type is purely informational.

All these informations are stored in the \texttt{docid} structure at the begining of each document. The revision number of the docid allows to keep track of the issued pdf. Each time a pdf is issued and shared, the revision number should be incremented so that the end reader can easily differentiate different issues of the same document.

Other examples:

\begin{center}
	\texttt{COILCHAIN\_220001TR\_TITLE}\\
	\texttt{COILCHAIN\_220000PRES\_TITLE}\\
\end{center}

See Section~\ref{sec:doctype} for a list of all suggested document types and 
code.

The document year should not change even if a new revision is made in another 
year. For example, if a document is published in year 2022 with identifier 
\texttt{COILCHAIN\_220001TR\_TITLE}, a subsequent revision in year 2023 should not modify its identifier, only the revision, internal to the document shall be updated.

\section{Standard Document Template}
This section describes the standard document templates that shall be used with 
the \LaTeX{} class to help authors write a compliant document. The 
class is used by adding \verb|\documentclass{coilchain}| at the beginning of 
the document.

\subsection{Document Identifier}\label{sec:docid}
The document id can be automatically generated using a \LaTeX{} command. For 
this document, the command is \verb|\docid{<params>}| where \verb|<params>| is 
a comma-separated list of key-value pairs. For this document, the command 
syntax to generate the document id \texttt{COILCHAIN\_220000DOC\_TITLE} is:
\begin{verbatim}
\docid{
	code = COILCHAIN,
	sequence = 0000,
	subcode = ,
	year = 2022,
	type = documentation,
	revision = 0.1}
\end{verbatim}

The available commands are:

\begin{tabular}{|l|l|}\hline
    \textbf{Key} & \textbf{Description} \\\hline

    \texttt{code} 
    & The document prefix code. \textbf{Default}: \texttt{COILCHAIN}.\\\hline
    
    \texttt{subcode} 
    & The document prefix subcode. \textbf{Optional}.\\\hline

    \texttt{type} 
    & The document type. \textbf{Required.}\\\hline
    
    
    \texttt{year} 
    & The document year. \textbf{Required.}\\\hline
    
    \texttt{sequence} 
    & The sequence number. \textbf{Required.}\\\hline

    \texttt{revision} 
    & The revision number. \textbf{Required.}\\\hline
\end{tabular}

\subsection{Document Type}\label{sec:doctype}
A few document types are already defined and should be used as much as 
possible.
Here is the list of the pre defined document types and their short code and use 
case.

\begin{tabularx}{\textwidth}{lX}
	\texttt{generalannouncement} (\texttt{GA}) & A document that will benefit 
	the whole 
	team and future members. \\
	\texttt{technicalreport} (\texttt{TR}) & A document that describe a process,
	progress, state or result of a technical or scientific research. \\
	\texttt{engineeringpolicy} (\texttt{EP}) & A document that establishes the 
	rules for designing new product and more general guidelines for engineering 
	related guidelines. \\
	\texttt{documentation} (\texttt{DOC}) & A document for documenting. 
	Technical report should be preferred whenever possible. \\
	\texttt{proposal} (\texttt{PR}) & A document proposing new documents, 
	policies...\\
	\texttt{procesverbal} (\texttt{PV}) & The minutes from a meeting with all 
	the decisions.\\
	\texttt{ordredujour} (\texttt{OJ}) & Ordre du jour for a meeting.\\
   	\texttt{weeklyreport} (\texttt{WR}) & Weekly report.
\end{tabularx}

The document title must match the document type and type code. One can set the 
document title using the command: \verb|\doctitle{<title>}|. If it is not set, 
an error will be raised.

\subsection{``Not For Publication'' Notice}\label{sec:notforpub}
If a document is for internal use only, a ``Not For Publication'' notice should 
be added to the document. In the \LaTeX{} template, one must add 
\verb|\notforpublication| in the preamble of the document. This command adds 
the following text on the first page of the document:
\begin{center}
	\textbf{\large Important notice:\\ this document is for 
		internal use only and is not intended\\ for distribution to the  
		public.}
\end{center}

The command also adds the \textbf{Not For Publication} notice in the footer of 
all subsequent pages.

\subsection{Watermark}
A watermark can be added using document class options. To add the 
\texttt{DRAFT} watermark, one must use the \verb|\wdraft| command at the beginning of the document. For 
\texttt{CONFIDENTIAL} watermark, the \verb|\wconfidential| watermark command can be 
used. The use of the \texttt{confidential} watermark should be reserved to 
sensitive information relative to the technology or strategy of coilchain, 
such as a document about domain names purchases.

\subsection{Title Page}
The first page of the document must have the coilchain logo, the document 
identifier, the document type, the document title and the last modification 
date, and an environment-friendly notice. If applicable, the authors may also 
be added to the title page. Authors should not be added in some cases, such as 
in a datasheet.

The authors are set using the \verb|\author{<author>}| command. If there are 
multiple authors, they must be sorted by the amount of work and 
alphabetically if no clear order can be established between two or more author. 
All authors of a document must have provided a significant enough amount of 
work to be included. The authoring information is important to inform the 
public who to contact about the publicly released document.
Examples:
\begin{verbatim}
   \author{Joe Dart}
   \author{Peter Pan}
   \author{Buzz Lightning}
\end{verbatim}

The document type in the title is set using  \verb|\doctitle{<type>}| as 
described in Section~\ref{sec:doctype}. The document identifier is set using 
the \verb|\docid{<params>}| command described in Section~\ref{sec:docid}. The 
document title is set using the \verb|\title{<title>}| command. Two lines 
titles are tolerated using \texttt{\textbackslash\textbackslash} e.g.
\texttt{\textbackslash{}title\{First line\textbackslash\textbackslash{}Second 
line\}}.

The title page also includes an introduction or abstract. This introduction 
text is added using the environment \verb|\introduction[<title>]|. The default 
title \textbf{Introduction} may be changed using the optional parameter:
\begin{verbatim}
    \begin{introduction}[Abstract]
        content...
    \end{introduction}
\end{verbatim}

An important ``Not For Publication'' notice shall be added on the first page if 
the document is not intended for distribution to the public. As described in 
Section~\ref{sec:notforpub}, one must use the \verb|\notforpublication| command 
to add the notice.

\subsection{Header and Footer}
The header and footer are automatically managed by the document class. They 
must include the following information:
\begin{itemize}
	\item Document identifier
	\item Date of last revision
	\item ``Not For Publication'' notice if need be
	\item Page number and total page number
	\item Current section (optional)
\end{itemize}

\subsection{Deprecated and Superseded Commands}
If the document is deprecated, use the \verb|\deprecated[<since>]| command 
where \verb|<since>| is an optional deprecation date. Here is the result with 
and without the deprecation date:

\textcolor{red}{\textbf{THIS DOCUMENT IS DEPRECATED SINCE JUNE 2017. FOR 
REFERENCE ONLY.}}\\
\textcolor{red}{\textbf{THIS DOCUMENT IS DEPRECATED. FOR REFERENCE ONLY.}}

If the document is superseded, use the \verb|\superseded[<by>]| command where 
\verb|<by>| is an optional document name. Here is the result with and 
without the superseding document name:

\textcolor{red}{\textbf{THIS DOCUMENT IS SUPERSEDED BY COILCHAIN\_XXXX. FOR 
		REFERENCE ONLY.}}\\
\textcolor{red}{\textbf{THIS DOCUMENT IS SUPERSEDED. FOR REFERENCE ONLY.}}

The red warnings are added on the title page.

\subsection{Supplementary material}
If supplementary material is required, such as Gerber files, the document must 
clearly state a way to obtain this supplementary material. For publicly 
available document, the supplementary material must be publicly available as 
well.


\subsection{Other Important Pages}
If applicable, the document may include a table of contents, a list of figure, 
a glossary, a list of tables, a list of equations... In addition, it may
include a legal notice page which is automatically handled by the \LaTeX{}
template. The legal notice page may be removed with the command 
\verb|\nodisclaimer|.

\subsection{Document Class Options}
Use the \verb|short| options for document with a small number of pages. No 
table of content must be included, nor any list of tables, figures... This will 
change the layout to be more suited for a few pages document, such as a weekly 
report.


\section{Version control}
The complete repository shall hold all the source necessary to retain the document information, thus all unecessary files are purposefully ommited from the repository.

\subsection{Directory usage}
All matters related to the class and style package are held in the \texttt{commons} directory.\\
All source files are held in the \texttt{src} directory.\\
All media specific to a single document are held in the \texttt{img} or \texttt{doc} directories inside the document directories. Within \texttt{img} a \texttt{src} directory can be used to store the vector graphics source files.\\

The \texttt{scripts} directory aims at storing usefull spripts for compiling, compressing or media management.\\

The \texttt{utils} directory aims at storing shared medias that are relevant to store in share root directory.

\subsection{Media guidelines}
All media should be vector .pdf file whenever possible.
Ideally the source file should be .svg vector graphics.
The color palette for the coilchain documentation is provided in the \texttt{palette} directory.
When a raster image is inserted make sure the pixel density \textbf{does not exceed 300dpi} to keep the remote directory lightweight and to produce a lighweight pdf file.

\subsection{.gitignore}
The gitignore will not allow misplaced .pdf file to be uploaded to the remote repository. Make sure the medias are in the appropriate location.

The compiled .pdf are kept local. Distribution of the files shall be performed by other mean than the Git repository to keep it lightweight and snappy.

\section{Tex configuration}
\subsection{MikTex config}

\figi{miktex_config.PNG}{fig:miktex}{0.7\linewidth}{MikTex PATH configuration}

\end{document}